\chapter{ZAKLJUČAK}
\label{chap:Poglavlje8}
Zaključno, u radu je metodologija implementacije PIV tehnologije podijeljena u nekoliko faza. Prikazan je problem evaluacije točnosti DPIV algoritama te su uz pomoć vlastitih umjetnih PIV snimki testirana četiri glavna DPIV algoritma implementirana u PIVlab sustav: DKK, obični DFT te dva DFT algoritma s ugrađenom tehnikom deformacije prozora: multi-linear DFT i multi-spline DFT. Ustanovljeno je kako multi-spline DFT najrobusnije reagira na razne vrste PIV snimki, prvenstveno zbog mogućnosti korelacije u više prolaza te zbog superiornijeg načina interpolacije novih dobivenih prozora ispitivanje (korištenje spline-a). Napisan je jednostavni vlastiti DPIV kros-korelacijski program koji ne sadrži nikakve dodatne tehnike obrade slike te su kvantitativni rezultati korelacijske analize uspoređeni s rezultatima iz PIVlab softvera koji ima mogućnost prethodnog i naknadnog poboljšavanja snimki. U napisanom DPIV kodu koji koristi MATLAB-ovu robusnu korelacijsku funkciju \textit{normxcorr2} i u PIVlab softveru za par analiziranih umjetno generiranih PIV snimki dobiveni su gotovo isti rezultati.
\par
Nadalje, uspješno je dizajniran i implementiran PIV sustav uz pomoć opreme dostupne na katedri fakulteta. Mjereno je istjecanje vode iz cijevi u spremnik pravokutnog oblika dimenzija $200\times200\times500$ mm. Voda je pumpana iz jednog spremnika u drugi, te prosječna brzina istrujavanja iz cijevi iznosi cca. 3 m/s. Napravljene su snimke uz pomoć Chronos 1.4 brze kamere pri 500 fps i rezoluciji snimanja od $1280\times1024$ pixela, korišten je niskobudžetni kontinuirani zeleni laser snage 50 mW koji na sebi ima ugrađenu optiku za generiranje laserske ravnine. U mjerenjima nisu korištene dodatne čestice markeri, nego je korištena voda iz slavine u kojoj je osvjetljeni sitni kamenac služio kao indikator gibanja. Dobivene snimke analizirane su u PIVlab softveru s uključenim opcijama pretprocesiranja snimki, te DFT tehnikom deformacije prozora u 3 prolaza kao DPIV algoritmom. Za veličinu prvog prozora ispitivanja od 128 px, te posljednjeg prozora od 32 pixela dobiveni su najkonkretniji rezultati. DPIV softver nije uspio izmjeriti brzine u zoni istjecanja iz cijevi zbog prevelikog utjecaja (3D) turbulencija. Naime, zbog prebrzog i nasumičnog gibanja na izlazu iz cijevi dolazi do  izvan-ravninskog gibanja čestica, te korelacijski algoritam ne može povezati parove čestica iz dvije uzastopne snimke. Dodatan je problem ne korištenje čestica markera koje bi  povećale korelacijski signal. Također, treba uzeti u obzir nisku kvalitetu korištenog lasera zbog male snage i neujednačenog osvjetljenja. Ipak, uz sve poteškoće prilikom mjerenja, u zoni izvan istjecanja iz cijevi dobiveni su intuitivno realniji rezultati koje i dalje treba uzeti sa dozom opreza. Na kraju je zaključeno kako je moguće korištenje implementirane opreme u daljnjem radu, ali uz uvjet mnogo boljoj optimizaciji parametara snimanja (npr. kontrola snage lasera, korištenje komercijalnih čestica markera, preciznija postavljanje mjerne opreme,...). Jedan od načina provjere dobivenih rezultata bio bi korištenje komercijalne PIV opreme, ili moguća i jednostavnija provjera polja brzina uz pomoć CFD simulacije.